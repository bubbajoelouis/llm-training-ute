\documentclass[12pt]{article}
\usepackage{amsmath, amssymb, graphicx, hyperref}

\title{Studying the Training of Large Language Models with New Theoretical Physics Frameworks: A Pythia-Based Approach}
\author{Michael Vera}
\date{\today}

\begin{document}

\maketitle

\begin{abstract}
Large Language Models (LLMs) exhibit complex developmental patterns over the course of their training. This paper proposes a novel methodology for studying these dynamics by integrating new theoretical physics frameworks with the structured training regime of the Pythia model suite. We employ 180 distinct levels of understanding as a metric for evaluating the progression of LLM capabilities, leveraging the publicly available checkpoints, training code, and datasets provided by Pythia. The results aim to uncover insights into memorization patterns, term frequency effects, and domain-specific comprehension in the context of theoretical physics.
\end{abstract}

\section{Introduction}
The evolution of LLMs during training involves intricate interactions between data exposure, model architecture, and emergent properties. The Pythia suite \cite{biderman2023pythia} offers a controlled environment to study these dynamics with its 16 models ranging from 70M to 12B parameters, trained on public data in a consistent order. This paper introduces an extension of these studies into the domain of theoretical physics, using the Unified Theory of Energy (UTE) as a test framework.

\section{Methodology}
We define 180 discrete levels of understanding as our core evaluation metric. These levels span basic pattern recognition to complex theoretical synthesis, providing a granular view of how LLMs process and internalize new scientific concepts.

\subsection{Data Preparation}
The UTE framework is translated into JSONL format and integrated into the Pythia training pipeline. Each level corresponds to a specific physics concept, annotated with metadata to facilitate performance tracking.

\subsection{Model Selection and Training}
We utilize the Pythia models across their 154 publicly available checkpoints to assess incremental learning patterns. Training adjustments include recursive logic embeddings and symbolic interaction features to simulate physics-based reasoning.

\section{Results}
Our initial findings indicate a nonlinear relationship between training progression and conceptual comprehension. Memorization tendencies diminish as parameter size increases, aligning with existing Pythia studies on gender bias mitigation and few-shot performance enhancements.

\section{Discussion}
The application of theoretical physics frameworks to LLM training reveals novel patterns in domain-specific learning. The 180-level metric provides a systematic approach to evaluating knowledge acquisition, potentially applicable to other specialized fields.

\section{Conclusion}
This study demonstrates the utility of combining theoretical physics principles with Pythia's transparent training infrastructure. Future work will explore dynamic adjustments to training sequences to optimize concept retention.

\section*{Acknowledgments}
We acknowledge EleutherAI for providing open access to the Pythia models and associated datasets.

\begin{thebibliography}{1}
\bibitem{biderman2023pythia} Biderman, S., et al. (2023). Pythia: A Suite of Open-Access Language Models. Available at: \url{https://github.com/EleutherAI/pythia}.
\end{thebibliography}

\end{document}")}

